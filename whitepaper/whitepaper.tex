\documentclass{article}
    % General document formatting
    \usepackage[margin=0.7in]{geometry}
    \usepackage[parfill]{parskip}
    \usepackage[utf8]{inputenc}
    
    % Related to math
    \usepackage{amsmath,amssymb,amsfonts,amsthm}

\title{Decentralized Compute Marketplace}
\author{Quphase}
\date{}


\begin{document}

\maketitle

\section{Introduction}
We don't aim to compete with AWS, we aim to create a marketplace for services like AWS.


\section{Math}
\subsection{Polynomial Ring $K[x]$}

We call a polynomial in $x$ over a field $K$ as $K[x]$ and define it as:

$$p = p_0 + p_1 x + p_2 x^2 + \hdots + p_m x^m$$

where $p_i \in K $ are the coefficients of polynomial $p$. And, as we dealt with in elementary school, we can add and multiply two polynomials together. For polynomials $p$ and $q$, we have addition $p+q$ and multiplication $p \times q$. The results of these operations leads to another polynomial.

A specific example of a polynomial are ones over the integers: 

$$\mathbb{R}[x]$$

Or over integers modulus $q$:

$$\mathbb{R}_q[x]$$



\subsection{Quotient Rings of Polynomial Rings}
One problem from the definition of polynomials, is that when we multiply two polynomials, we get a degree of a polynomial that is larger than the ones we started with. To solve this, we can work under a polynomial modulus. A specific example are polynomials of the following form:

$$\mathbb{Z}_q[x] / (x^n + 1)$$

Now addition and multiplication over the polynomials are done \texttt{mod} $x^n + 1$ (i.e. divide the result by $x^n + 1$ and take the remainder). The insures that all of our polynomials have degree $d < n$



\end{document}
